\documentclass[titlepage]{article}

\usepackage[margin=1in]{geometry}

% Tables and stopping them from displaying in a different section
\usepackage{booktabs}
\usepackage[section]{placeins}

% for inserting images into the document, setting file path, and allowing rotation of inserted images 
\usepackage{graphicx}
\graphicspath{ {./images/} }
\usepackage{rotating}
\usepackage[table]{xcolor}
% mostly just for putting text in math equations
\usepackage{amsmath}
% for aligning the text to the left
\usepackage[document]{ragged2e}

% for inserting hyperlinks in the document, use \url{url} or \href{url}{text}
\usepackage{hyperref}
\usepackage{siunitx}
\usepackage{caption}
\usepackage{multirow}
\usepackage[export]{adjustbox}

\begin{document}

\title{\textbf{Lab 2: Electric Field and Potential}}
\author{Report by: Zachary Pouska and  Natalie Tran \\ \\ 
PHYS 236 | Fall 2022}
\date{Date performed: 09/21/2022}

	\maketitle
	\section{Purpose}
	To study the relationship betwween electric field and the electric potential difference associated with it.
	\section{Theory}	
	The relationship between the electric field and electric potential difference will follow the equation \(\Delta V = -\int_{a}^{b} \vec{E} \cdot ds\), which simplified is \(\Delta V = \frac{k_{e}q}{r}\). This means electric potential will have a opposite yet linear relationship with the electric field, while having an inverse relationship with distance.
	\section{Experiment Analysis}
	\section{Procedure}
	\section{Data and Graphs}
	\section{Results}
	\section{Questions}
	\section{Conclusion}
	
\end{document}
