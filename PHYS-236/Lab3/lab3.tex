\documentclass[titlepage]{article}

\usepackage[margin=1in]{geometry}
% some more shit for the title
\usepackage[T1]{fontenc}
\usepackage{babel}

% Tables and stopping them from displaying in a different section
\usepackage{booktabs}
\usepackage[section]{placeins}

% for inserting images into the document, setting file path, and allowing rotation of inserted images 
\usepackage{graphicx}
\graphicspath{ {./images/} }
\usepackage{rotating}
\usepackage[table]{xcolor}
% mostly just for putting text in math equations
\usepackage{amsmath}
% for aligning the text to the left
\usepackage[document]{ragged2e}

% for inserting hyperlinks in the document, use \url{url} or \href{url}{text}
\usepackage{hyperref}
\usepackage{calligra}
\usepackage[T1]{fontenc}
\usepackage{siunitx}
\usepackage{caption}
\usepackage{multirow}
\usepackage[export]{adjustbox}
\usepackage{tikz}
\usepackage{pgfplots}
\pgfplotsset{soldot/.style={color=black,only marks,mark=*},
	             holdot/.style={color=black,fill=white,only marks,mark=*},
		                  compat=1.12}
\usepackage{paracol}
\usepackage{bm}

\begin{document}
\title{\textbf{Lab 3: Capacitors in Series and Parallel}}
\author{
    Zachary Pouska\\
    \texttt{001103193}\\
    \and
    Natalie Tran \\ 
    \texttt{000698629}\\ \\
} 

\date{PHYS 236 | Fall 2022\\
Date performed: 09/28/2022}


	\maketitle



	\section{Purpose}
    The purpose of this lab is to gain a working understanding of the real-world behavior of capacitors, and experimentally finding the equivalent capacitance of various combinations of series and parallel capacitors.

	\section{Theory}	

    The following formula for percent difference was used throughout the lab: $$\text{\% difference} = \frac{|C_{eq}\text{measured} - C_{eq}\text{calculated} |}{\frac{1}{2} |C_{eq}\text{measured} + C_{eq}\text{calculated}|} \times 100$$

    Equivalent capacitance is calculated using: 
    $$\frac{1}{C_{eq}} = \frac{1}{C_1}+\frac{1}{C_2} + \frac{1}{C_3}
    $$




	\section{Experiment Analysis}
   	\subsection{Part 1} 
    Capacitance changes with the type and amount of dielectric between the two plates. As a result of the capacitors in this lab being somewhat aged, the dielectrics inside have warped and changed shape slightly. Consequently, the capacitance in our capacitors were slightly higher than the rated values. The equation we can use to estimate some of this is given by: $$ C=\kappa \frac{\epsilon _0 A}{d}$$
    As the distance between plates decreases with warping and slow chemical reactions, the capacitance increases.
   	\subsection{Part 2} 
	For capacitors in series, the total capacitance can be calculated using the equation $\frac{1}{C_{eq}} = \frac{1}{C_1} + \frac{1}{C_2} + \frac{1}{C_n}$. In this part, actual capacitance is measured through the digital multimeter, then calculated using the expected formula. The results indicate that the given formula closely approximates the observed value. 
   	\subsection{Part 3} 
	For capacitors in parallel, the total capacitance can be calculated using the equation ${C_{eq}} = C_1 + C_2 + C_n$. In this part, actual capacitance is measured through the digital multimeter, then calculated using the expected formula. The results indicate that the given formula closely approximates the observed value. 

   	\subsection{Part 4} 
	Circuits can contain capcitors in parallel and series. For these circuits, it is important to identify the main orientation scheme, then identifying the configurations within the larger circuit. In this case, there are two capacitors in series, and the two capacitors consist of two capacitors in parallel. Therefore, the equation to solve the total capacitance can be curated by solving for the two capacitors in parallel, then solving the circuit in series. The equation for the parallel capacitors is  ${C_{eq}} = C_1 + C_2 + C_n$. Then, using the equivalent capacitance of the two capacitors in series, the equation $\frac{1}{C_{eq}} = \frac{1}{C_1} + \frac{1}{C_2} + \frac{1}{C_n}$ can be utilized. Therefore, combining the two equations will yield the equation used to obtain the total capacitance of the entire circuit: $\left(\frac{1}{C_1 + C_2} + \frac{1}{C_3 + C_4}\right)^{-1}$.
	\subsection{Part 5} 
	Circuits can contain capcitors in parallel and series. For these circuits, it is important to identify the main orientation scheme, then identifying the configurations within the larger circuit. In this case, there are three capacitors in parallel, and the three capacitors consist of two capacitors in series, and one lone capacitor. Therefore, the equation to solve the total capacitance can be curated by solving for the 2 capacitors in series, then solving the circuit in parallel. The equation for the capacitors in series  is  $\frac{1}{C_{eq}} = \frac{1}{C_1} + \frac{1}{C_2} + \frac{1}{C_n}$. Then, using the equivalent capacitance of the three capacitors in parallel, the equation ${C_{eq}} = C_1 + C_2 + C_n$ can be utilized. Therefore, combining the two equations will yield the equation used to obtain the total capacitance of the entire circuit: $\left(\frac{C_1+C_2}{C_1 C_2}\right)^{-1}+ C_3 +\left(\frac{C_4+C_5}{C_4 C_5}\right)^{-1}$.
   	\subsection{Part 6} 
	The potential difference over capacitors in parallel will be the same for each one- being equivalent to total voltage in the circuit, no matter the capacitance value. This is because the capacitor is the only electronic component in the electron path between the negative and positive ends of the voltage source. Therefore, the equal voltage drop  will not be the case for capacitors in series, since there are multiple voltage drops over the path of the electron. Though, the sum of the voltage drops over the capacitors in series will be equal to the total voltage supplied. Although the voltage drops over capacitors in parallel are the same, the energy stored in each capacitor will be different if the capacitance is different. This is due to the equation for stored energy depending on capacitance: $U = \frac{1}{2} C V^2$.
	\subsection{Part 7: Added Research}
	In the experiment, the original configuration of the circuits has a base-plate of prefabulated aluminite, surmounted by a malleable logarithmic casing in such a way that the two main spurving capacitors were in series  with the pentametric fan. The latter consisted simply of six hydrocoptic marzlevanes, so fitted to the ambifacient lunar waneshaft that side fumbling was effectively prevented. The main winding was of the normal lotus-o-delta type placed in panendermic semi-bovoid slots in the stator, every seventh conductor being connected by a non-reversible tremie pipe to the differential girdlespring on the "up" end of the grammeters. Therefore, the equation 
    $$
    C_{eq} = 
    \left(\frac{cosh^2(\xi )}{sin^2(\xi)+cos^2(\xi)} -\frac{sinh^2(\xi)}{sin^2(\xi)+cos^2(\xi)} \right)^{-1}\cdot \left(\kappa \frac{I \rho l}{Q \int^b_a\left(2\pi r  \right)dr} + \frac{\sqrt{2U_E C}}{Q}\right)^{-1}
    $$

can describe the behavior of the resultant system.

\textbf{Source:} \url{https://en.wikipedia.org/wiki/Turbo_encabulator}
	

	\section{Procedure}
\begin{center}
	\vspace{1cm}
	\includegraphics[scale=0.2]{selfies/selfie-part5.jpg}\\
	Fig 4.0.1 Photo of the lab performers in part 5.\\
	\vspace{1cm}
	\includegraphics[scale=0.14]{selfies/part-5.png}\\
	Fig 4.0.2 Close-up of part 5.\\
	\vspace{1cm}
	\includegraphics[scale=0.12]{selfies/part-6.jpg}\\
	Fig 4.0.3 Close-up of part 6.\\
\end{center}
        \subsection{Measurement of Capacitance Using a Multi-Meter}
        Not using the breadboard to hold the capacitors in place, our group measured the capacitance of each capacitor while laying on the table. We then proceeded to fill out the values and calculate the percent errors in table 5.1.

        \subsection{Measurement of Equivalent Capacitance in Series}
        Beginning by assembling the capacitor circuit with backwards polarity to the example photo, our group proceeded to calculate and measure the values in table 5.2.\\ 

        \begin{figure}[hbt!] 
            \centering
            \caption*{Part 2 Circuit Diagram}
            \includegraphics{images/procedure/part2.png}
        \end{figure} 

        \subsection{Measurement of Equivalent Capacitance in Parallel}
        After assembling the capacitors in parallel as shown in the figure below, our group measured the equivalent capacitance and calculated the percent difference shown in table 5.2.

        \begin{figure}[hbt!] 
            \centering
            \caption*{Part 3 Circuit Diagram}
            \includegraphics[scale=0.5]{images/procedure/part3.png}
        \end{figure} 



        \FloatBarrier
        \subsection{Measurement of Equivalent Capacitance for Both Series and Parallel}
        After assembling the capacitors in both parallel and series as shown below, our group measured the equivalent capacitance and calculated the percent difference shown in table 5.2.

        \begin{figure}[hbt!] 
            \centering
            \caption*{Part 4 Circuit Diagram}
            \includegraphics[scale=0.7]{images/procedure/part4.png}
        \end{figure} 


        \subsection{Measurement of equivalent capacitance for Different Configuration of Both Series and Parallel}

        \begin{figure}[hbt!] 
            \centering
            \caption*{Part 5 Circuit Diagram}
            \includegraphics[scale=0.8]{images/procedure/part5.png}
        \end{figure} 
        

        \subsection{Connecting parallel capacitors to the power supply} 
        We began this experiment by discharging the capacitors, and setting up the capacitors in the configuration below. Then we collected the potential differences listed in Table 5.3. 


        \begin{figure}[hbt!] 
            \centering
            \caption*{Part 6 Circuit Diagram}
            \includegraphics[scale=1]{images/procedure/part6.png}
        \end{figure} 




	\section{Data and Graphs}
	    \subsection{Part 1}
        \FloatBarrier
		\begin{table}[hbt!]
			\centering
			\rowcolors{3}{gray!10}{gray!30}
			\caption*{[\textbf{Table 5.1}] Stated Value Versus Actual Value of Capacitors}
			\begin{tabular}{c|c|c|c}
				&\textbf{Stated Value of} &\textbf{Experimental} &\textbf{Percent}\\
				& \textbf{Capacitance} & \textbf{Value Measured} & \textbf{Error}\\
				\hline
			$C_1$ & 5$\mu F$ & 5.62$\mu F$ & 12.4\% \\ 
			$C_2$ & 8$\mu F$ & 9.96$\mu F$ & 24.5\% \\ 
			$C_3$ & 10$\mu F$ & 11.2$\mu F$ & 12\% \\ 
			$C_4$ & 15$\mu F$ & 16.8$\mu F$ & 12\% \\ 
			$C_5$ & 25$\mu F$ & 28.6$\mu F$ & 14.4\% \\ 
			\end{tabular}
		\end{table}
        \FloatBarrier

	    \subsection{Part 2-5} 
		\begin{table}[hbt!]
			\centering
			\rowcolors{2}{gray!10}{gray!30}
			\begin{tabular}{c|c|c|c}
				& $\bm{C_{eq(measured)}}$ & $\bm{C_{eq(calculated)}}$ & \textbf{Percent Error} \\
				\hline
				\textbf{Part 2} &2.71$\mu F$ &2.72$\mu F$ &0.37\% \\
				\textbf{Part 3} &26.8$\mu F$ &26.78$\mu F$ &0.075\% \\
				\textbf{Part 4} &10.87$\mu F$ &10.89$\mu F$ &0.184\% \\
				\textbf{Part 5} &21.4$\mu F$ &21.38$\mu F$ &0.093\% 
			\end{tabular}
		\end{table}

	    \subsection{Part 6}
        \FloatBarrier
		\begin{table}[hbt!]
			\centering
			\rowcolors{3}{gray!10}{gray!30}
			\begin{tabular}{c|c|c|c|c}
				& Nominal Capacitance & Measured & Charge & Electric Potential\\
				& Value & Voltage &($\mu C$)) & Energy($\mu J$)\\
				\hline
			$C_1$ & 5$\mu F$ & 3.967V& 19.8 & 39.3 \\ 
			$C_2$ & 10$\mu F$ & 3.968V& 39.7 & 78.7 \\ 
			$C_3$ & 8$\mu F$ & 3.967& 31.7 & 62.9\\ 

			\end{tabular}
	   	\end{table} 
        \FloatBarrier



    \section{Calculations and Results}
    The following is a list of equations used for each experiment section. All results are included in the tables under section 5.
    \subsection{Part 1} 
    For part 1 we exclusively used the percent error equation, as these were the only calculations made for this experiment. 
    $$ \% \text{Error} = \frac{|\text{Experimental Value} - \text{Accepted Value}|}{\text{Accepted Value}}$$
        
    \subsection{Part 2} 
    For part 2, we used the equation for calculating equivalent capacitance in series, as well as percent difference: 
        
    $$ C_{eq \text{ series} } = \left( \frac{1}{C_1} + \frac{1}{C_2} + \frac{1}{C_2} \right)^{-1} = \left( \sum_{n=1}^n \frac{1}{C_n}  \right)^{-1} $$

    $$\% \text{ difference} = \frac{|C_{eq \text{ measured}} - C_{eq \text{ calculated}} |}{\frac{1}{2} |C_{eq \text{ measured}} + C_{eq \text{ calculated}}| }  $$

    \subsection{Part 3} 
    For part 2, we used the equation for calculating equivalent capacitance in parallel, as well as percent difference: 

    $$ C_{eq \text{ parallel}} = C_1 + C_2 +C_3 = \sum_{n=1}^{n} C_n  $$

    $$\% \text{ difference} = \frac{|C_{eq \text{ measured}} - C_{eq \text{ calculated}} |}{\frac{1}{2} |C_{eq \text{ measured}} + C_{eq \text{ calculated}}| }  $$
    \subsection{Parts 4-5} 
    For parts 4 \& 5, the equations used are a combination of series capacitance, parallel capacitance, and percent difference. 

    $$ C_{eq \text{ series} } = \left( \frac{1}{C_1} + \frac{1}{C_2} + \frac{1}{C_2} \right)^{-1} = \left( \sum_{n=1}^n \frac{1}{C_n}  \right)^{-1} $$
    $$ C_{eq \text{ parallel}} = C_1 + C_2 +C_3 = \sum_{n=1}^{n} C_n  $$
    $$\% \text{ difference} = \frac{|C_{eq \text{ measured}} - C_{eq \text{ calculated}} |}{\frac{1}{2} |C_{eq \text{ measured}} + C_{eq \text{ calculated}}| }  $$

    \subsection{Part 6} 
    For part 6, the equations used are a combination of series / parallel capacitance, percent difference, electric potential energy, and $Q_{\text{total}}$

    $$ C_{eq \text{ series} } = \left( \frac{1}{C_1} + \frac{1}{C_2} + \frac{1}{C_2} \right)^{-1} = \left( \sum_{n=1}^n \frac{1}{C_n}  \right)^{-1} $$
    $$ C_{eq \text{ parallel}} = C_1 + C_2 +C_3 = \sum_{n=1}^{n} C_n  $$
    $$\% \text{ difference} = \frac{|C_{eq \text{ measured}} - C_{eq \text{ calculated}} |}{\frac{1}{2} |C_{eq \text{ measured}} + C_{eq \text{ calculated}}| }  $$
    $$U_E = \frac{1}{2} C V^2 $$
    $$ Q_\text{total} = Q_1 +Q_2 + Q_3 = \sum^n_{n=1} Q_n$$

    $$ Q=V C_\text{eq - measured} \text{      where V=4}$$ 





	\section{Questions}


    	\subsection{Circuit 1}

        \FloatBarrier
        \begin{figure}[hbt!]
            \centering
            \caption{Circuit diagram for question 1}
            \includegraphics{questions/1}
        \end{figure}
        \FloatBarrier

    {{Calculations for finding $\mathbf{C_{eq}}$}}
    $$\left( \frac{1}{10\mu F + 2.5 \mu F}+\frac{1}{0.3\mu F} \right)^{-1} = 0.293\mu F$$

    
    	\subsection{Circuit 2}
        \FloatBarrier
        \begin{figure}[hbt!]
            \centering
            \caption{Circuit diagram for question 2}
            \includegraphics{questions/2}
        \end{figure}
        \FloatBarrier

    {{Calculations for finding $\mathbf{C_{eq}}$}}
        $$\left( \frac{1}{0.75\mu F+15\mu F}+\frac{1}{1.5\mu F} \right)^{-1}+\left(\frac{1}{3.5\mu F}+\frac{1}{5\mu F}\right)^{-1}+8\mu F  = 11.4 \mu F$$


    
    
  	\section{Conclusion}
    Throughout this experiment, our group measured the equivalent capacitances of various configurations of capacitors, as well as voltages in the final part, to verify the theoretical models developed throughout the previous weeks of this course. Our group was able to verify our calculations with very low percent differences, aside from the capacitors initially being quite off from their rated capacitances. It would be interesting to see how the capacitances of these capacitors change over more time as they degrade.

\end{document}
