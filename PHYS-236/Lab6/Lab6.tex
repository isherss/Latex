\documentclass[titlepage]{article}

\usepackage[margin=1in]{geometry}
% some more shit for the title
\usepackage[T1]{fontenc}
\usepackage{babel}

% Tables and stopping them from displaying in a different section
\usepackage{booktabs}
\usepackage[section]{placeins}

% for inserting images into the document, setting file path, and allowing rotation of inserted images 
\usepackage{graphicx}
\graphicspath{ {./images/} }
\usepackage{rotating}
\usepackage[table]{xcolor}
% mostly just for putting text in math equations
\usepackage{amsmath}
% for aligning the text to the left
\usepackage[document]{ragged2e}

% for inserting hyperlinks in the document, use \url{url} or \href{url}{text}
\usepackage{hyperref}
\usepackage{calligra}
\usepackage[T1]{fontenc}
\usepackage{siunitx}
\usepackage{caption}
\usepackage{multirow}
\usepackage[export]{adjustbox}
\usepackage{tikz}
\usepackage{pgfplots}
\pgfplotsset{soldot/.style={color=black,only marks,mark=*},
	             holdot/.style={color=black,fill=white,only marks,mark=*},
		                  compat=1.12}
\usepackage{paracol}

\begin{document}
\title{\textbf{Lab 6: RC Circuits}}
\author{
    Zachary Pouska\\
    \texttt{001103193}\\
    \and
    Natalie Tran \\ 
    \texttt{000698629}\\ \\
} 

\date{PHYS 236 | Fall 2022\\
Date performed: 11/02/2022}


	\maketitle



	\section{Purpose}
	\section{Theory}	
	\section{Experiment Analysis}


    $$V_c = \varepsilon \left( 1- e^{-\frac{t}{\tau}} \right)  | \div $$
    



	\section{Procedure}




	\section{Data and Graphs}
	\subsection{Part 1}
	\subsection{Part 2} 
	\subsection{Part 3}
	\section{Results}
	\section{Questions}


	\subsection{Part 1}

	\subsection{Part 2}

    \subsection{Part 3}
	
    \subsection{Part 4}
    \begin{enumerate}
        \item Do you obtain the same values for the voltage across the resistor and capacitor? Explain.\\ 
            \textbf{Yes! They are in parallel, so the potential difference across each should be the same. If the potential difference wasn’t equal, that wouldn’t make sense, as measuring the potential difference across each one is essentially connecting the multimeter to the same point in the circuit, assuming 0 resistance in the wires.}
        \item Is the current across the resistor zero? Explain.\\ 
            \textbf{No. In the case of the resistor and capacitor being in series, as the capacitor fills up, it blocks the current flow through the resistor. In this configuration, as the capacitor charges more current is simply diverted through the resistor instead of through the capacitor.}



    \end{enumerate}

	\section{Conclusion}

\end{document}
