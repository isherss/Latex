\documentclass[titlepage]{article}

\usepackage[margin=1in]{geometry}
% some more shit for the title
\usepackage[T1]{fontenc}
\usepackage{babel}

% Tables and stopping them from displaying in a different section
\usepackage{booktabs}
\usepackage[section]{placeins}

% for inserting images into the document, setting file path, and allowing rotation of inserted images 
\usepackage{graphicx}
\graphicspath{ {./images/} }
\usepackage{rotating}
\usepackage[table]{xcolor}
% mostly just for putting text in math equations
\usepackage{amsmath}
% for aligning the text to the left
\usepackage[document]{ragged2e}

% for inserting hyperlinks in the document, use \url{url} or \href{url}{text}
\usepackage{hyperref}
\usepackage{calligra}
\usepackage[T1]{fontenc}
\usepackage{siunitx}
\usepackage{caption}
\usepackage{multirow}
\usepackage[export]{adjustbox}
\usepackage{tikz}
\usepackage{pgfplots}
\pgfplotsset{soldot/.style={color=black,only marks,mark=*},
	             holdot/.style={color=black,fill=white,only marks,mark=*},
		                  compat=1.12}
\usepackage{paracol}

\begin{document}
\title{\textbf{Lab 4: Resistivity of Nickel Chromium Wire and Use of the Wheatstone Bridge Circuit}}
\author{
    Zachary Pouska\\
    \texttt{001103193}\\
    \and
    Natalie Tran \\ 
    \texttt{000698629}\\ \\
    \and
    Joseph Pancho\\
    \texttt{002550975} \\ \\
} 

\date{PHYS 236 | Fall 2022\\
Date performed: 10/10/2022}


	\maketitle



	\section{Purpose}
	In this lab, we measured the resistance of a nickel chromium wire 
	and calculated the resistivity $\rho$. We then built a Wheatstone 
	bridge to find the resistances of individual capacitors.

	\section{Theory}
	Using nickel chromium wire $(80\%~Ni- 20\%~Cr)$, we will apply the 
	equations for calculating resistivity $\rho$. \\
	~\\
	For a given wire resistivity $\rho$, length $L$, and cross-sectional 
	area $A$, the resistance $R$, is given by: \\
	\[
		R=\frac{\rho L}{A}
	\]
	Solving for $\rho$, the above equation is re-written as:
	\[
		\rho = \frac{RA}{L}
	\]	 
	Verifying the untis for $\rho$: 
	\[
		\rho=\frac{\Omega m^2}{m}=\Omega m	
	\]
	\section{Experiment Analysis}
    



	\section{Procedure}




	\section{Data and Graphs}
	\subsection{Part 1}
	\subsection{Part 2} 
	\subsection{Part 3}
	\section{Results}
	\section{Questions}


	\subsection{Part 1}

	\subsection{Part 2}

    \subsection{Part 3}
	
    \subsection{Part 4}

	\section{Conclusion}

\end{document}
